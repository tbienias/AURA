%%%%%%%%%%%%%%%%%%%%%%%%%%%%%%%%%%%%%%%%%
% Thin Sectioned Essay
% LaTeX Template
% Version 1.0 (3/8/13)
%
% This template has been downloaded from:
% http://www.LaTeXTemplates.com
%
% Original Author:
% Nicolas Diaz (nsdiaz@uc.cl) with extensive modifications by:
% Vel (vel@latextemplates.com)
%
% License:
% CC BY-NC-SA 3.0 (http://creativecommons.org/licenses/by-nc-sa/3.0/)
%
%%%%%%%%%%%%%%%%%%%%%%%%%%%%%%%%%%%%%%%%%

%----------------------------------------------------------------------------------------
%	PACKAGES AND OTHER DOCUMENT CONFIGURATIONS
%----------------------------------------------------------------------------------------

\documentclass[a4paper, 11pt]{article} % Font size (can be 10pt, 11pt or 12pt) and paper size (remove a4paper for US letter paper)

\usepackage[protrusion=true,expansion=true]{microtype} % Better typography
\usepackage{graphicx} % Required for including pictures
\usepackage{wrapfig} % Allows in-line images

\usepackage{mathpazo} % Use the Palatino font
\usepackage[T1]{fontenc} % Required for accented characters
\linespread{1.05} % Change line spacing here, Palatino benefits from a slight increase by default

\makeatletter
\renewcommand\@biblabel[1]{\textbf{#1.}} % Change the square brackets for each bibliography item from '[1]' to '1.'
\renewcommand{\@listI}{\itemsep=0pt} % Reduce the space between items in the itemize and enumerate environments and the bibliography

\renewcommand{\maketitle}{ % Customize the title - do not edit title and author name here, see the TITLE block below
\begin{flushright} % Right align
{\LARGE\@title} % Increase the font size of the title

\vspace{50pt} % Some vertical space between the title and author name

{\large\@author} % Author name
\\\@date % Date

\vspace{40pt} % Some vertical space between the author block and abstract
\end{flushright}
}

\usepackage[utf8]{inputenc} % Umlaute unter Mac werden automatisch gesetzt
\usepackage[T1]{fontenc} % Zeichenencoding
\usepackage[ngerman]{babel} 

%----------------------------------------------------------------------------------------
%	TITLE
%----------------------------------------------------------------------------------------

\title{\textbf{Techischer Bericht: Sprint 1}\\ % Title
MPSE Hololens\\ bei Frau Prof. Dr. Hergenröther} % Subtitle

\author{\textsc{Tim Bienias \\ Hakan Durgel \\ Ibrahim Cinar} % Author
\\{\textit{Hochschule Darmstadt}}} % Institution

\date{\today} % Date

%----------------------------------------------------------------------------------------

\begin{document}

\shorthandoff{"}


\maketitle % Print the title section

%----------------------------------------------------------------------------------------
%	ABSTRACT AND KEYWORDS
%----------------------------------------------------------------------------------------

%\renewcommand{\abstractname}{Summary} % Uncomment to change the name of the abstract to something else

%\begin{abstract}
%Morbi tempor congue porta. Proin semper, leo vitae faucibus dictum, metus mauris lacinia lorem, ac congue leo felis eu turpis. Sed nec nunc pellentesque, gravida eros at, porttitor ipsum. Praesent consequat urna a lacus lobortis ultrices eget ac metus. In tempus hendrerit rhoncus. Mauris dignissim turpis id sollicitudin lacinia. Praesent libero tellus, fringilla nec ullamcorper at, ultrices id nulla. Phasellus placerat a tellus a malesuada.
%\end{abstract}

%\hspace*{3,6mm}\textit{Keywords:} lorem , ipsum , dolor , sit amet , lectus % Keywords

%\vspace{30pt} % Some vertical space between the abstract and first section

%----------------------------------------------------------------------------------------
%	ESSAY BODY
%----------------------------------------------------------------------------------------
\section*{Assetmanagement}

\section*{Aufgabenstellung}
In diesem Sprint galt es zwei Backlog-Items zu bearbeiten. Diese setzten sich wie folgt zusammen:
\begin{enumerate}
    \item Repository für Textdateien und andere Dateien erstellen. Muss Versionierung und Pushing bzw. Pulling
    unterstützen.
    \item Rekonfiguration des erstellten Repositories, sodass Asset-Dateien in einer speicherplatzsparenden Art 
    gespeichert werden.
\end{enumerate}

%------------------------------------------------

\section*{1. Gerneral Repository}
Zuerst wurde Research betrieben, indem nach Versionierungssystemen gesucht wurde, welche die im Backlog spezifizierten 
Eigenschaften aufweisen. In die finale Auswahl sind hierbei Git und Perforce gekommen, da diese hochmodernen 
Anforderungen genügen und als Industriestandard anerkannt sind.
\\ Letztlich fiel die Entscheidung für Git aus, da dieses Versionierungssystem die im Backlog spezifizierten 
Anforderungen unterstützt und im Gegensatz zu Perforce geringeren Aufwand bzgl. Einrichtung etc. mit sich bringt.
Das Repository wurde erstellt und mittels einer Komprimierung letztlich in die Projekt-Versionsverwaltung 
hochgeladen.

%------------------------------------------------

\section*{2. Asset Repository}
Da das Asset-Format nicht spezifiziert wurde und sich dazu keine Dokumente finden ließen,
definierte unsere Gruppe, dass das Asset-Format "FBX 6.1 ASCII" ist, welches von führenden 
3D-Content-Creation-Programmen, wie z.B. Blender, unterstützt wird.
Da dieses Format 3D-Assets in Klartext speichert, ist somit keine Rekonfiguration des Repositories, zwecks Kompression, 
notwendig. \\
Wir definierten zwei Expermimente. Diese bestanden zum einen darin, dass ein Würfel exportiert wird und zum anderen 
darin, dass der Szene ein zusätzlicher Würfel hinzugefügt wird um eine Änderung der Asset-Datei zu simulieren.
Nachfolgend eine Aufstellung der angestellten Berechnungen:
\begin{itemize}
    \item Leeres Repository: 56883 Bytes
    \item Würfel in FBX 6.1 ASCII: 78589 Bytes
    \item Hypothese Repository Größe: 56883 + 78589 = 135472 Bytes
    \item Tatsächliche Repository Größe: 76867 Bytes
\end{itemize}
Wie aus den Berechnungen hervorgeht, erreicht unser Team eine Kompressionsrate von nahezu 50\%. Da dies eine relativ
hohe Rate darstellt, befanden wir diese für gut genug.\\
Zweites Experiment mit Änderung der exportierten Datei:
\begin{itemize}
    \item Ausgangsgröße nach Experiment 1: 76867 Bytes
    \item Geänderte FBX-Datei mit 2 Würfeln: 87425 Bytes
    \item Tatsächliche Repository Größe nach Experiment 2: 97489 Bytes
\end{itemize}
Daraus folgt, dass reale Unterschied zwischen den beiden FBX-Dateien auf dem Filesystem 87425 - 78589 = 8836 Bytes ist.
Der Größenunterschied des Repositories nach Experiment 2 beträgt 97489 - 76867 = 20622 Bytes. \\
Somit folgern wir daraus, dass nicht die Tatsächliche FBX-Datei zwei mal gespeichert wird, sondern dass lediglich ein
Diff mit mutmaßlichem Verwaltungs-Overhead im Repository abgelegt wird. Dies führt dazu, dass die spezifizierten 
Anforderungen nun vollends erfüllt sind.

%------------------------------------------------

\section*{Retrospektive}
Sprint 1 wurde gemäß den spezifizierten Anforderungen erfolgreich abgeschlossen. Es gab keine Probleme und das Team 
ist bereit für Sprint 2.

%----------------------------------------------------------------------------------------
%	BIBLIOGRAPHY
%----------------------------------------------------------------------------------------


%\bibliographystyle{unsrt}

%\bibliography{sample}

%----------------------------------------------------------------------------------------

\end{document}